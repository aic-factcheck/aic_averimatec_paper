%!TEX ROOT=../emnlp2023.tex

% show figures/pipeline.png


\section{Introduction}
The challenge of automated fact verification has been studied extensively in previous works~\cite{10.1162/tacl_a_00454,akhtar-etal-2025-2nd,schlichtkrull-etal-2024-automated}, most commonly modelled as an NLP task with textual inputs.
With public discourse moving increasingly to social media, the task fact-checkers face, however, often goes beyond just text and language.
An important example of this phenomenon are the image-text claims, whose veracity depends not only on the textual statement itself, but also on the contents of images that come with it, whether they are authentic or edited, and whether the images are presented in the right context.

To facilitate the automation of this type of fact-checking,~\citealt{cao2025averimatecdatasetautomaticverification} publishes the \averitec{} dataset, collecting hundreds of reference image-text factchecks from human annotators, announcing the \averitec{} shared task late 2025, to establish its state of the art.

%\vspace{-.5em}
\begin{minipage}{\linewidth}
    \centering
    %\includegraphics[width=\linewidth]{figures/AVERIMATEC.drawio.png}
    \includegraphics[width=\linewidth]{figures/pipeline2026.pdf}
    \captionof{figure}{Our image-text fact-checking pipeline used in CTU AIC AVerImaTeC submission, adapted from~\citealt{ullrich-drchal-2025-aic}. System is described in detail in section~\ref{sec:system2026}.}
    \label{fig:pipeline2026}
\vspace{1em}
\end{minipage}


With this paper, we introduce our 3rd place \averitec{} shared-task system which aims to give a strong baseline for image-text fact checking using easy-to-reproduce modules and affordable running costs, with a single query to multimodal LLM per claim and a single RIS request for each image attached to it.
Our pipeline performs RAG with two retrieval modules, one retrieving relevant documents from offline knowledge using vector search, other retrieving documents that better contextualize the claim images using RIS -- Google Lens, in our case.
Our system is visualised in figure~\ref{fig:pipeline2026} and detailed in section~\ref{sec:system2026}.