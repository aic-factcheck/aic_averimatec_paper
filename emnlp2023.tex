% This must be in the first 5 lines to tell arXiv to use pdfLaTeX, which is strongly recommended.
\pdfoutput=1
% In particular, the hyperref package requires pdfLaTeX in order to break URLs across lines.

\documentclass[11pt]{article}

% Remove the "review" option to generate the final version.
\usepackage{EMNLP2023}

% Standard package includes
\usepackage{times}
\usepackage{latexsym}
\usepackage{float}

% For proper rendering and hyphenation of words containing Latin characters (including in bib files)
\usepackage[T1]{fontenc}
% For Vietnamese characters
% \usepackage[T5]{fontenc}
% See https://www.latex-project.org/help/documentation/encguide.pdf for other character sets

% This assumes your files are encoded as UTF8
\usepackage[utf8]{inputenc}

% This is not strictly necessary, and may be commented out.
% However, it will improve the layout of the manuscript,
% and will typically save some space.
\usepackage{microtype}

% This is also not strictly necessary, and may be commented out.
% However, it will improve the aesthetics of text in
% the typewriter font.
\usepackage{inconsolata}
% includegraphics
\usepackage{graphicx}
%listings
\usepackage{listings}
\usepackage{dirtytalk}
\usepackage{enumitem}
\usepackage{graphicx}
\usepackage{array}
\usepackage{booktabs}
%review macros
\usepackage{xcolor}
\newcommand{\review}[1]{{\color{black}#1}}

%pandas tables
\usepackage{{booktabs}}
% Commands
\newcommand{\todo}[1]{{\color{red}\colorbox{yellow}{\textbf{TODO: }}#1}}
\newcommand{\averitec}{AVerImaTeC}
\newcommand{\evr}{Ev\textsuperscript{2}R}
\newcommand{\supp}{Supported}
\newcommand{\reff}{Refuted}
\newcommand{\nei}{Not enough evidence}
\newcommand{\conf}{Conflicting evidence/Cherrypicking}
\makeatletter
\newcommand\footnoteref[1]{\protected@xdef\@thefnmark{\ref{#1}}\@footnotemark}
\makeatother

% If the title and author information does not fit in the area allocated, uncomment the following
%
%\setlength\titlebox{<dim>}
%
% and set <dim> to something 5cm or larger.

\title{AIC CTU@AVerImaTeC: dual-retriever RAG for image-text fact checking}

% Author information can be set in various styles:
% For several authors from the same institution:
% \author{Author 1 \and ... \and Author n \\
%         Address line \\ ... \\ Address line}
% if the names do not fit well on one line use
%         Author 1 \\ {\bf Author 2} \\ ... \\ {\bf Author n} \\
% For authors from different institutions:
% \author{Author 1 \\ Address line \\  ... \\ Address line
%         \And  ... \And
%         Author n \\ Address line \\ ... \\ Address line}
% To start a seperate ``row'' of authors use \AND, as in
% \author{Author 1 \\ Address line \\  ... \\ Address line
%         \AND
%         Author 2 \\ Address line \\ ... \\ Address line \And
%         Author 3 \\ Address line \\ ... \\ Address line}

\author{Herbert Ullrich \\
AI Center @ CTU FEE\\
Charles Square 13\\
Prague, Czech Republic\\
\texttt{ullriher@fel.cvut.cz} \\\And
Jan Drchal \\
AI Center @ CTU FEE\\
Charles Square 13\\
Prague, Czech Republic\\
\texttt{drchajan@fel.cvut.cz} \\}

\begin{document}
%{\makeatletter\acl@finalcopytrue
  \maketitle
%}
\begin{abstract}
In this paper, we present our 3rd place system in the AVerImaTeC shared task, which combines our last year's retrieval-augmented generation (RAG) pipeline with a reverse image search (RIS) module.
Despite its simplicity, our system delivers competitive performance with a single multimodal LLM call per fact-check at just \$0.013 on average using GPT5.1 via OpenAI batch API.
Our system is also easy to reproduce and tweak, consisting of only three decoupled modules -- a textual retrieval module based on similarity search, an image retrieval module based on API-accessed RIS, and a generation module using GPT5.1 -- which is why we suggest it as an accesible starting point for further experimentation.
We publish its code\footnote{\url{https://github.com/heruberuto/AVerImaTec\_Shared\_Task}} and prompts, as well as our vector stores and insights into the scheme's running costs and directions for further improvement.

\end{abstract}

%%%%%%%%%%%%%%%%%%%%%%%%%%%%%%%%%%%%
% inputs
%!TEX ROOT=../emnlp2023.tex

% show figures/pipeline.png


\section{Introduction}
The challenge of automated fact verification has been studied extensively in previous works~\cite{10.1162/tacl_a_00454,akhtar-etal-2025-2nd,schlichtkrull-etal-2024-automated}, most commonly modelled as an NLP task with textual inputs.
With public discourse moving increasingly to social media, the task fact-checkers face, however, often goes beyond just text and language.
An important example of this phenomenon are the image-text claims, whose veracity depends not only on the textual statement itself, but also on the contents of images that come with it, whether they are authentic or edited, and whether the images are presented in the right context.

To facilitate the automation of this type of fact-checking,~\citealt{cao2025averimatecdatasetautomaticverification} publishes the \averitec{} dataset, collecting hundreds of reference image-text factchecks from human annotators, announcing the \averitec{} shared task late 2025, to establish its state of the art.

%\vspace{-.5em}
\begin{minipage}{\linewidth}
    \centering
    %\includegraphics[width=\linewidth]{figures/AVERIMATEC.drawio.png}
    \includegraphics[width=\linewidth]{figures/pipeline2026.pdf}
    \captionof{figure}{Our image-text fact-checking pipeline used in CTU AIC AVerImaTeC submission, adapted from~\citealt{ullrich-drchal-2025-aic}. System is described in detail in section~\ref{sec:system2026}.}
    \label{fig:pipeline2026}
\vspace{1em}
\end{minipage}


With this paper, we introduce our 3rd place \averitec{} shared-task system which aims to give a strong baseline for image-text fact checking using easy-to-reproduce modules and affordable running costs, with a single query to multimodal LLM per claim and a single RIS request for each image attached to it.
Our pipeline performs RAG with two retrieval modules, one retrieving relevant documents from offline knowledge using vector search, other retrieving documents that better contextualize the claim images using RIS -- Google Lens, in our case.
Our system is visualised in figure~\ref{fig:pipeline2026} and detailed in section~\ref{sec:system2026}.
%!TEX ROOT=../emnlp2023.tex

\section{System description}
\label{sec:system2026}

We adapted a system from~\citealt{ullrich-drchal-2025-aic} which extends on top of~\citealt{ullrich-etal-2024-aic}.
The cited paper describes the system in detail, with ablation studies and justifications of each step.
Our pipeline, depicted in Figure~\ref{fig:pipeline2026}, consists of precomputation, retrieval, and generation modules:

\begin{enumerate}[label=\roman*.]  % First-level: i., ii., iii.
\item Text-based retrieval module
\begin{enumerate}[label=\arabic*.]  % Second-level: 1., 2., 3.
    \item ~
\end{enumerate}
\item Image-based retrieval module
\begin{enumerate}[label=\arabic*.]  % Second-level: 1., 2., 3.
    \item ~
\end{enumerate}
\item Evidence \& label generation module
\begin{enumerate}[label=\arabic*.]  % Second-level: 1., 2., 3.
    \item ~
\end{enumerate}
\end{enumerate}

The main differences between this year's AIC \averitec{} system, opposed to last year's AIC AVeriTeC system, are the omission of knowledge store pruning in the precomputation step\footnote{The precomputed vector stores were required to be independent on claim text in \averitec{}.}, and, importantly, the choice of LLM.
\subsection{Model and parameter choices}
\label{sec:choices}
To produce our submission in the \averitec{} shared task, the following choices were made to deploy the pipeline from section~\ref{sec:system2026}:

\texttt{mxbai-embed-large-v1}~\cite{li-li-2024-aoe,emb2024mxbai} is used for the vector embeddings, and the maximum chunk size is set to 2048 characters, considering its input size of 512 tokens and a rule-of-thumb coefficient of 4 characters per token to exploit the full embedding input size and produce the smallest possible vector store size without neglecting a significant proportion of knowledge store text.

\texttt{FAISS}~\cite{douze2024faiss,johnson2019billion} index is used as the vector database engine, due to its simplicity of usage, exact search feature and quick retrieval times (sub-second for a single \averitec{} test claim).

$l=10, k=40, \lambda=0.75$ are the parameters we use for the MMR reranking, meaning that 40 chunks are retrieved, 10 sources are yielded after MMR-diversification, and the tradeoff between their similarity to the claim and their diversity is 3:1 in favour of the source similarity to the claim (explained in more detail in~\citealt{ullrich-etal-2024-aic}). 

%\input{src/classification}
%!TEX ROOT=../emnlp2023.tex

\section{Results and analysis}
\label{nothink}


\begin{table}[h]
\centering
\begin{tabular}{l
>{\centering\arraybackslash}p{.7cm} 
>{\centering\arraybackslash}p{.7cm} 
>{\centering\arraybackslash}p{.7cm} 
>{\centering\arraybackslash}p{.7cm}}
{\small{\textbf{System}}} &
\rotatebox{70}{\textbf{\footnotesize{Question Score}}} &
\rotatebox{70}{\textbf{\footnotesize{Evidence Score}}} &
\rotatebox{70}{\textbf{\footnotesize{Verdict Accuracy}}} &
\rotatebox{70}{\textbf{\footnotesize{Justification Score}}} \\
\hline
{\small{HUMANE}}        & {0.89} & {0.54} & {0.55} & {0.56} \\
{\small{ADA-AGGR}}      & 0.37 & 0.46 & 0.54 & 0.43 \\
{\textit{\small{AIC CTU (ours)} }}       & \textit{0.81} & \textit{0.33} & \textit{0.35} & \textit{0.30} \\
{\small{XxP}}           & 0.39 & 0.27 & 0.26 & 0.20 \\
{\small{teamName}}      & 0.66 & 0.23 & 0.26 & 0.22 \\
{\small{REVEAL}}        & 0.63 & 0.28 & 0.24 & 0.13 \\
{\small{fv}}            & 0.29 & 0.16 & 0.16 & 0.13 \\
\hline
{\small{Baseline}}      & 0.55 & 0.17 & 0.11 & 0.13 \\
\end{tabular}
\caption{System leaderboard showing performance metrics on \averitec{} test-split. Our system described in section~\ref{sec:system2026} is highlighted with \textit{italics}.}
\label{tab:leaderboard}
\end{table}

The final \averitec{} leaderboard is shown in table~\ref{tab:leaderboard}. Our system achieves a combined verdict score\footnote{Proportion of claims with a correct verdict \textit{and} an evidence score of at least 0.3 at the same time, see~\citealt{cao2025averimatecdatasetautomaticverification}.} of 0.35, with a near-SOTA question score of 0.81, mean evidence score of 0.35, and a justification score of 0.3.
Metrics are based on Ev2R~\cite{akhtar2024ev2r} recall scores with LLM as a judge.

While our system does not reach the very state of the art, it significantly outperforms the iterative agentic baseline~\cite{cao2025averimatecdatasetautomaticverification} and majority of other systems across the board, scoring a solid 3rd place.
To reveal directions for future improvements, we proceed to study what its main pitfalls are using the leaderboard metrics and our own reproductions of \averitec{} dev-split metrics.


\subsection{Bottlenecks}
\label{sec:bottlenecks}
\begin{table}[h]
\centering
\begin{tabular}{l
>{\centering\arraybackslash}p{.7cm} 
>{\centering\arraybackslash}p{.7cm} 
>{\centering\arraybackslash}p{.7cm} 
>{\centering\arraybackslash}p{.7cm}}
{\small{\textbf{Evidence format}}} &
\rotatebox{70}{\textbf{\footnotesize{Question Score}}} &
\rotatebox{70}{\textbf{\footnotesize{Evidence Score}}} &
\rotatebox{70}{\textbf{\footnotesize{Verdict Accuracy}}} &
\rotatebox{70}{\textbf{\footnotesize{Justification Score}}} \\
\hline
{\small{Answer only}}   & \textbf{0.86} & 0.27 & 0.31 & 0.28 \\
{\textit{\small{Question + Answer}}}   & \textit{0.84} & \textit{0.33} & \textit{\textbf{0.39}} & \textit{0.31} \\
{\small{Declarative evidence}}   & 0.82 & \textbf{0.35} & 0.38 & \textbf{0.32} \\
\end{tabular}
\caption{Ablation study tweaking the evidence generation format from section~\ref{sec:system2026}, iii. Scheme used in final submission is in italics.}
\label{tab:ablation}
\end{table}

Looking at our standing in the leaderboard from table~\ref{tab:leaderboard}, the main bottleneck appears to be our system's \textit{evidence score}, computed using Ev2R recall. Despite the question score shows promising 81\% our lack in evidence score then propagates further to the verdict and justification scores as well.
Part of this problem could be attributed to our system's legacy evidence format geared more towards AVeriTeC 1 and 2 shared tasks -- an \textit{evidence} is generated as a QA pair, of a question fact-checker would ask themselves during the task, and an answer they would arrive to, grounded in an URL-referred source, whereas in \averitec{} evaluation scheme, the evidence is a self-contained declarative sentence with pointers to relevant images.

Table~\ref{tab:ablation} lists three approaches we took to address this discrepancy -- in our first approach, we disregarded this entirely and only listed the generated answers as the \averitec{} evidence, in our second approach -- which is also the one we have submitted to the final leaderboard (table~\ref{tab:leaderboard}) -- we have concatenated the question and answer together to obtain each evidence string, appending a \texttt{[IMG\_1]} tag and a base64-encoded image in metadata where an image-related source was used. 

To see whether this can be improved upon, we have also experimentally implemented a 3rd approach, referred to as \say{declarative evidence} in table~\ref{tab:ablation}, in which we have directly prompted the LLM to generate a self-contained declarative evidence text with pointers to used images.
Although this approach was experimental and not free of its own glitches (resulting in a malformed image pointers and \texttt{[IMG\_1]} tag being erroneously used in other generic fields, such as justification and questions), it shows promissing results, surpassing our \textit{Question+Answer} approach by encouraging 2\% in the evidence score, even before adjusting its prompt to iron out the glitches.

Another bottleneck could be possible discrepancies in our image-evidence usage -- looking closer at the ablation study in table~\ref{tab:ablation}, the \say{Answer only} approach stays too close behind its more advanced alternatives.
This finding raises concerns, since the answer-only approach does not use \textit{any} \texttt{[IMG\_1]} tags, yet per~\citealt{cao2025averimatecdatasetautomaticverification}, 53.9\% of the \averitec{} evidence should be annotated using reverse image search, with 1.6\% using the image itself as the answer. 
This is to be investigated in future works, as even a small discrepancy in the way our system presents its image sources and how the \averitec{} evaluator assumes to receive them may have a tremendous impact on the final score.


\subsection{Cost analysis}
The scheme from section~\ref{sec:system2026} uses a single RIS request per claim image (one claim may feature multiple images, but the vast majority features exactly 1 image in \averitec{}) -- using Serper, this search comes at a cost of 3 credits, totalling at \$0.003 using the least-discounted bulk pricing (\$50 for 50K Serper credits).

The markdown scraping was performed using the Firecrawl API, which at its hobby tier charges \$0.006 per scraped page, with 20,000 free scraping tasks for education emails -- in the worst case scenario of multiple claim images in a single claim, each with full 9 RIS results older than the claim date that can be scraped\footnote{Which is not usually the case, as at least some proportion of results typically come from Meta's scraping-protected social media} and no discount, this amounts to \$0.05 per image.
To avoid this cost, however,  we suggest using a free scraper instead, such as the Trafilatura library which was used to produce the \averitec{} offline knowledge stores and our system does not show any noticeable problems ingesting its outputs.

The Generation module LLM results were computed using the OpenAI Batch API, with GPT-5.1 as the backbone model.
On average, 11K completion input tokens were given to the model and 1150 tokens of output were generated per \averitec{} claim using our system from section~\ref{sec:system2026}, at an average cost of \$0.013 per claim.
%\input{src/software.tex}
%!TEX ROOT=../emnlp2023.tex

\section{Conclusion}


\subsection{Future works}
\begin{enumerate}
    \item Integrate an agentic "should image-based evidence be retrieved?" approach into the pipeline to see if it raises the score
\end{enumerate}

%%%%%%%%%%%%%%%%%%%%%%%%%%%%%%%%%%%%

\section*{Limitations}
Our pipeline is not meant to be relied upon nor to replace a human fact-checker, but rather to assist an informed user. It gives sources for both the textual and image-text evidence and proposes labels for further questioning. Further hallucination could be spotted within the generated justification, however, the system is not meant to be used as an oracle. The current prompting and text-retrieval model assume English input, and the MLLM-backbone used for the shared task (although interchangeable) is GPT5.1, which is a black box model with limited reproducibility and considerable carbon costs. 
Refuted class is massively overrepresented in the \averitec{} dataset (95\% of train-claims and 78\% of text-claims), making the accuracy-based \averitec{} score computed over \averitec{} test set a problematic metric for systems used in the wild.

\section*{Ethics statement}
Our pipeline is an extension of our already existing last year submission all original authors agreed with.
The system was build specifically for the~\averitec~shared task and reflects the biases of its annotators, for more information on this, we suggest the original \averitec{} paper~\cite{cao2025averimatecdatasetautomaticverification}.
\section*{Acknowledgements}
We would like to thank Tomáš Mlynář for providing insights into multi-modal retrieval systems.

This article was created with the state support of the Ministry of Industry and Trade of the Czech Republic, project no. Z220312000000, within the National Recovery Plan Programme.
The access to the computational infrastructure of the OP VVV funded project CZ.02.1.01/0.0/0.0/16\_019/0000765 ``Research Center for Informatics'' is also gratefully acknowledged.



% Entries for the entire Anthology, followed by custom entries
\bibliography{anthology,custom}
\bibliographystyle{acl_natbib}

\appendix

%!TEX ROOT=../emnlp2023.tex


\lstset{
    language={},
    basicstyle=\ttfamily\footnotesize\linespread{0.9}, % Smaller font with less spacing
    keywordstyle=\color{blue}\bfseries,
    commentstyle=\color{green!50!black}\itshape,
    stringstyle=\color{orange},
    numberstyle=\tiny\color{gray},
    numbers=none, % Line numbers on the left
    stepnumber=1, % Line numbers for every line
    numbersep=5pt, % Space between line numbers and code
    tabsize=4, % Size of tabs
    showstringspaces=false, % Don't show spaces in strings
    breaklines=true, % Line wrapping
    breakatwhitespace=true,
    frame=lines, % Add a frame around the code
    captionpos=b, % Caption at the 
    breakindent=1em,
}
\begin{figure*}
    \section{System prompt}
    \label{appendix_sec:system2026_prompt}
    \begin{lstlisting}[breaklines=true, language={}, frame=single, caption={Our fact-checking system prompt to be used with MLLM, feeding the \averitec{} claim text and images into its multimodal user message. Three dots represent omitted repeating parts of the prompt. Adapted for multimodal scenario from~\citealt{ullrich-drchal-2025-aic}.}, label={lst:llm_system_prompt}]
You are a professional fact checker of image-text claims, formulate up to 10 questions that cover all the facts needed to validate whether the factual statement (in User message) is true, false, uncertain or a matter of opinion. The claim consists of a textual statement and {image_count} images associated with the claim. The claim was made by {author} on {date} via {medium}. Each question has one of four answer types: Boolean, Extractive, Abstractive and Unanswerable using the provided sources.
After formulating Your questions and their answers using the provided sources, You evaluate the possible veracity verdicts (Supported claim, Refuted claim, Not enough evidence, or Conflicting evidence/Cherrypicking) given your claim and evidence on a Likert scale (1 - Strongly disagree, 2 - Disagree, 3 - Neutral, 4 - Agree, 5 - Strongly agree). Ultimately, you note the single likeliest veracity verdict according to your best knowledge.
The facts must be coming from the sources listed below. The first {k} sources was retrieved using textual search and the rest was retrieved using reverse image search (google lens). The sources are numbered - sources 1 through {k} are related to the claim text,  sources 11-19 were retrieved for the first user image, 21-29 to the second etc. You may therefore assume that each of the image-based sources was published alongside a picture similar to the respective user image. 
---
## Source ID: 1 [url]
[context before]
[page content]
[context after]
...
---
## Image Source ID: 11 (related to user image 1, Title : [title], date:[page_date], url: [url], image url: [img_url])
[content]
...
---
## Output formatting
Please, you MUST only print the output in the following output format:
```json
{
 "questions":
     [
         {"question": "<Your first question>", "answer": "<The answer to the Your first question>", "source": "<Single numeric source ID backing the answer for Your first question>", "answer_type":"<The type of first answer>"},
         {"question": "<Your second question>", "answer": "<The answer to the Your second question>", "source": "<Single numeric Source ID backing the answer for Your second question>", "answer_type":"<The type of second answer>"}
     ],
 "claim_veracity": {
     "Supported": "<Likert-scale rating of how much You agree with the 'Supported' veracity classification>",
     "Refuted": "<Likert-scale rating of how much You agree with the 'Refuted' veracity classification>",
     "Not Enough Evidence": "<Likert-scale rating of how much You agree with the 'Not Enough Evidence' veracity classification>",
     "Conflicting Evidence/Cherrypicking": "<Likert-scale rating of how much You agree with the 'Conflicting Evidence/Cherrypicking' veracity classification>"
 },
  "veracity_verdict": "<The suggested veracity classification for the claim>",
  "verdict_justification": "<A brief justification of the veracity verdict>"
}
```
---
## Few-shot learning
You have access to the following few-shot learning examples for questions and answers.:

### Question examples for claim "{example["claim"]}" (verdict {example["gold_label"]})
"question": "{question}", "answer": "{answer}", "answer_type": "{answer_type}"
...
    \end{lstlisting}
\end{figure*}
%\include{src/appendix_b_think}
%\include{src/appendix_b_opensource}
%\include{src/appendix_c_errors}

\end{document}

